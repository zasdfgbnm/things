%% LyX 2.3.6.1 created this file.  For more info, see http://www.lyx.org/.
%% Do not edit unless you really know what you are doing.
\documentclass[english]{article}
\usepackage[T1]{fontenc}
\usepackage[latin9]{inputenc}
\usepackage{amstext}
\usepackage{babel}
\begin{document}
1. This manuscript is highly related to its reference {[}7{]}. For
example, there are theorems looking very similar: Theorem 2.4 of this
manuscript vs. Theorem 2.4 in {[}7{]}, and Theorem 2.5 of this manuscript
vs. Theorem 2.6 in {[}7{]}. The author should elaborate more on what
{[}7{]} does and what is the original contribution of this manuscript
compared to {[}7{]}, instead of just ``... by adapting the work of
Sharma et al. {[}7{]}'' and ``For $k=1$, we obtain the bounds of
{[}7{]}''.

2. In the abstract of this article, the author says, ``using a projection
operator and a vector function of the eigenvalues...'', and in Definition
2.1, the author defined $\mathrm{P}$. But I don't see how these are
related to the rest of this article. I think both this sentence in
the abstract and Definition 2.1 should be removed from this article.

3. In Lemma 2.3, equation (1) is written with the language of inner
product. But in the proof of this lemma, equation (8) is obtained.
Although (1) and (8) are equivalent, (8) seems to be more useful in
the proof of later theorems. So I think the author should eliminate
the language of inner products and use (8) to replace (1) in Lemma
2.3.

4. The symbol $\left\langle \cdot,\cdot\right\rangle $ is defined
in Lemma 2.3, but it is first used in Definition 2.2, which appears
before Lemma 2.3. The author should move the definition of $\left\langle \cdot,\cdot\right\rangle $
into Definition 2.2 if the author really wants to use the language
of inner product in Definition 2.2 and Lemma 2.3. But according to
my previous point, I think the language of inner product is not a
good fit for this manuscript, so I prefer the author to remove it.

5. Page 3 Equation (5) should be 
\[
\ldots=\left(f\left(\lambda_{j}\right)-m\right)+\sum_{i=1,i\neq j}^{n}\left(f\left(\lambda_{i}\right)-m\right)
\]
instead of
\[
\ldots=\left(f\left(\lambda_{i}\right)-m\right)+\sum_{i=1,i\neq j}^{n}\left(f\left(\lambda_{i}\right)-m\right)
\]

6. How to compute $m$ efficiently? The author needs to be explicit
about this. This is important because if $m$ had to be computed from
$f\left(\lambda_{i}\right)$, then the bounds in this manuscript would
not be interesting because it was not computationally cheaper than
the exact value. I believe $m$ could be computed efficiently as $m=\frac{1}{n}\text{trace}\left[f\left(A\right)\right]$,
and I think the author should explicitly mention this in this paper.

7. Page 4, the equation after equation (13), should be $m-f\left(\lambda_{n}\right)\geq\ldots$
instead of $m-f\left(\lambda_{n}\right.\geq\ldots$

8. I think the author should list the results from the other bounds
that they compare to, especially {[}8{]}, in Table 1. {[}8{]} is a
submitted paper that is not accepted yet, so I don't have access to
it. Adding numbers from {[}8{]} will help me better understand how
this work improves {[}8{]}.
\end{document}
